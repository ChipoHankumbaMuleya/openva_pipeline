%% Generated by Sphinx.
\def\sphinxdocclass{report}
\documentclass[letterpaper,12pt,english]{sphinxmanual}
\ifdefined\pdfpxdimen
   \let\sphinxpxdimen\pdfpxdimen\else\newdimen\sphinxpxdimen
\fi \sphinxpxdimen=.75bp\relax

\PassOptionsToPackage{warn}{textcomp}
\usepackage[utf8]{inputenc}
\ifdefined\DeclareUnicodeCharacter
 \ifdefined\DeclareUnicodeCharacterAsOptional
  \DeclareUnicodeCharacter{"00A0}{\nobreakspace}
  \DeclareUnicodeCharacter{"2500}{\sphinxunichar{2500}}
  \DeclareUnicodeCharacter{"2502}{\sphinxunichar{2502}}
  \DeclareUnicodeCharacter{"2514}{\sphinxunichar{2514}}
  \DeclareUnicodeCharacter{"251C}{\sphinxunichar{251C}}
  \DeclareUnicodeCharacter{"2572}{\textbackslash}
 \else
  \DeclareUnicodeCharacter{00A0}{\nobreakspace}
  \DeclareUnicodeCharacter{2500}{\sphinxunichar{2500}}
  \DeclareUnicodeCharacter{2502}{\sphinxunichar{2502}}
  \DeclareUnicodeCharacter{2514}{\sphinxunichar{2514}}
  \DeclareUnicodeCharacter{251C}{\sphinxunichar{251C}}
  \DeclareUnicodeCharacter{2572}{\textbackslash}
 \fi
\fi
\usepackage{cmap}
\usepackage[T1]{fontenc}
\usepackage{amsmath,amssymb,amstext}
\usepackage{babel}
\usepackage{times}
\usepackage[Bjarne]{fncychap}
\usepackage{sphinx}

\usepackage{geometry}

% Include hyperref last.
\usepackage{hyperref}
% Fix anchor placement for figures with captions.
\usepackage{hypcap}% it must be loaded after hyperref.
% Set up styles of URL: it should be placed after hyperref.
\urlstyle{same}
\addto\captionsenglish{\renewcommand{\contentsname}{Contents:}}

\addto\captionsenglish{\renewcommand{\figurename}{Fig.}}
\addto\captionsenglish{\renewcommand{\tablename}{Table}}
\addto\captionsenglish{\renewcommand{\literalblockname}{Listing}}

\addto\captionsenglish{\renewcommand{\literalblockcontinuedname}{continued from previous page}}
\addto\captionsenglish{\renewcommand{\literalblockcontinuesname}{continues on next page}}

\addto\extrasenglish{\def\pageautorefname{page}}

\setcounter{tocdepth}{1}



\title{openva\_pipeline Documentation}
\date{Oct 25, 2018}
\release{0.0.0.9000}
\author{Jason Thomas, Samuel J. Clark, and Martin Bratschi}
\newcommand{\sphinxlogo}{\vbox{}}
\renewcommand{\releasename}{Release}
\makeindex

\begin{document}

\maketitle
\sphinxtableofcontents
\phantomsection\label{\detokenize{index::doc}}


The \sphinxhref{https://github.com/verbal-autopsy-software/openva\_pipeline}{OpenVA Pipeline} automates the
processing of verbal autopsy (VA) data from an \sphinxhref{https://docs.opendatakit.org/aggregate-intro/}{ODK Aggregate server}, through the \sphinxhref{https://cran.r-project.org/web/packages/openVA/index.html}{openVA} library from the
\sphinxhref{https://cran.r-project.org/}{R} statistical software, and ending with a
DHIS2 server (with the \sphinxhref{https://github.com/SwissTPH/dhis2\_va\_draft}{VA program}).  A friendly graphical user
interface to R is also provided via an \sphinxhref{https://github.com/verbal-autopsy-software/shinyVA}{RShiny app} and
vignette.


\chapter{Software Requirements}
\label{\detokenize{software:software-requirements}}\label{\detokenize{software::doc}}
The following software is required by the openVA pipeline (note: installation instructions are found on a different page)
\begin{itemize}
\item {} \begin{description}
\item[{\sphinxhref{https://www.python.org/downloads/}{Python 3.5.0 (or higher)}}] \leavevmode\begin{itemize}
\item {} 
\sphinxhref{https://pypi.python.org/pypi/pip}{PIP} (tool for installing Python packages)

\end{itemize}

\end{description}

\item {} 
\sphinxhref{http://openjdk.java.net}{OpenJDK} or
\sphinxhref{http://www.oracle.com/technetwork/java/javase/downloads/jdk8-downloads-2133151.html}{Java JDK 7 or 8},

\item {} 
\sphinxhref{https://cran.r-project.org}{R} (OpenVA Pipeline was tested on Version 3.4.3)

\item {} 
\sphinxhref{https://www.sqlite.org}{SQLite3}

\item {} 
\sphinxhref{https://github.com/sqlcipher/sqlcipher}{SQLCipher}

\item {} 
\sphinxhref{https://github.com/opendatakit/briefcase/releases}{ODK Briefcase} (OpenVA Pipeline was tested on Version v1.10.1)

\end{itemize}

It is also useful to install \sphinxhref{https://github.com/sqlitebrowser/sqlitebrowser/blob/master/BUILDING.md}{DB Browser}  for SQLite.  This
optional tool is useful for configuring the SQLite Database.


\chapter{Installation Guide}
\label{\detokenize{install:installation-guide}}\label{\detokenize{install::doc}}
The following instructions guide the installation of the OpenVA Pipeline on Ubuntu 16.04 operating system.
\begin{quote}

\begin{sphinxadmonition}{note}{Note:}
To make the installation process easier, all of the required software can be installed by downloading and running the bash script \sphinxhref{https://raw.githubusercontent.com/D4H-CRVS/OpenVA\_Pipeline/master/install\_software.sh}{install\_software.sh} located in the main folder of the OpenVA\_Pipeline repository.
\end{sphinxadmonition}
\end{quote}
\begin{enumerate}
\item {} 
Install Python, OpenJDK, \sphinxstylestrong{R}, SQLite3, SQLCipher, and Git by typing the following commands at a terminal prompt (indicated by \$)

\fvset{hllines={, ,}}%
\begin{sphinxVerbatim}[commandchars=\\\{\}]
\PYGZdl{} sudo apt update
\PYGZdl{} sudo apt install python3\PYGZhy{}pip python3\PYGZhy{}dev openjdk\PYGZhy{}8\PYGZhy{}jre r\PYGZhy{}base r\PYGZhy{}cran\PYGZhy{}rjava sqlite3 libsqlite3\PYGZhy{}dev sqlcipher libsqlcipher\PYGZhy{}dev git \PYGZhy{}y
\end{sphinxVerbatim}

\item {} 
Download the \sphinxhref{https://github.com/opendatakit/briefcase/releases}{ODK-Briefcase-v1.10.1.jar} file to the same folder where \sphinxstyleemphasis{pipeline.py}
is located.

\item {} 
In a terminal, start \sphinxstylestrong{R} by simply typing \sphinxcode{\sphinxupquote{R}} at the prompt, or use \sphinxcode{\sphinxupquote{sudo R}} for system-wide installation of
the packages.  The necessary packages can be installed (with an internet connection) using the following command:

\fvset{hllines={, ,}}%
\begin{sphinxVerbatim}[commandchars=\\\{\}]
\PYG{o}{\PYGZgt{}} \PYG{n}{install}\PYG{o}{.}\PYG{n}{packages}\PYG{p}{(}\PYG{n}{c}\PYG{p}{(}\PYG{l+s+s2}{\PYGZdq{}}\PYG{l+s+s2}{openVA}\PYG{l+s+s2}{\PYGZdq{}}\PYG{p}{,} \PYG{l+s+s2}{\PYGZdq{}}\PYG{l+s+s2}{CrossVA}\PYG{l+s+s2}{\PYGZdq{}}\PYG{p}{)}\PYG{p}{,} \PYG{n}{dependencies}\PYG{o}{=}\PYG{n}{TRUE}\PYG{p}{)}
\end{sphinxVerbatim}

Note that \sphinxcode{\sphinxupquote{\textgreater{}}} in the previous command indicates the \sphinxstylestrong{R} prompt (not part of the actual command).  This command will
prompt the user to select a CRAN mirror (choose a mirror close to your geographic location).  After the installation
of the packages has been completed, you can exit \sphinxstylestrong{R} with the following command:

\fvset{hllines={, ,}}%
\begin{sphinxVerbatim}[commandchars=\\\{\}]
\PYG{o}{\PYGZgt{}} \PYG{n}{q}\PYG{p}{(}\PYG{l+s+s1}{\PYGZsq{}}\PYG{l+s+s1}{no}\PYG{l+s+s1}{\PYGZsq{}}\PYG{p}{)}
\end{sphinxVerbatim}

\item {} 
Python is pre-installed with Ubuntu 16.04, but additional packages and modules are needed, which can be installed
with the following commands at a terminal:

\fvset{hllines={, ,}}%
\begin{sphinxVerbatim}[commandchars=\\\{\}]
\PYGZdl{} pip3 install \PYGZhy{}\PYGZhy{}upgrade pip \PYGZhy{}\PYGZhy{}user
\PYGZdl{} hash \PYGZhy{}d pip3
\PYGZdl{} pip3 install \PYGZhy{}\PYGZhy{}upgrade setuptools \PYGZhy{}\PYGZhy{}user
\PYGZdl{} pip3 install requests pysqlcipher3 \PYGZhy{}\PYGZhy{}user
\end{sphinxVerbatim}

Note: the first command: \sphinxcode{\sphinxupquote{pip3 install -{-}upgrade pip -{-}user}} will produce a warning message:

\fvset{hllines={, ,}}%
\begin{sphinxVerbatim}[commandchars=\\\{\}]
\PYG{n}{You} \PYG{n}{are} \PYG{n}{using} \PYG{n}{pip} \PYG{n}{version} \PYG{l+m+mf}{8.1}\PYG{o}{.}\PYG{l+m+mi}{1}\PYG{p}{,} \PYG{n}{however} \PYG{n}{version} \PYG{l+m+mf}{10.0}\PYG{o}{.}\PYG{l+m+mi}{1} \PYG{o+ow}{is} \PYG{n}{available}\PYG{o}{.}
\PYG{n}{You} \PYG{n}{should} \PYG{n}{consider} \PYG{n}{upgrading} \PYG{n}{via} \PYG{n}{the} \PYG{l+s+s1}{\PYGZsq{}}\PYG{l+s+s1}{pip install \PYGZhy{}\PYGZhy{}upgrade pip}\PYG{l+s+s1}{\PYGZsq{}} \PYG{n}{command}\PYG{o}{.}
\end{sphinxVerbatim}

However, after running the command \sphinxcode{\sphinxupquote{hash -d pip3}}, the command \sphinxcode{\sphinxupquote{pip3 -{-}version}} shows that version 10.0.1 is
indeed installed.

\item {} 
Install DB Browser for SQLite with the commands

\fvset{hllines={, ,}}%
\begin{sphinxVerbatim}[commandchars=\\\{\}]
\PYGZdl{} sudo apt install build\PYGZhy{}essential git\PYGZhy{}core cmake libsqlite3\PYGZhy{}dev qt5\PYGZhy{}default qttools5\PYGZhy{}dev\PYGZhy{}tools libsqlcipher\PYGZhy{}dev \PYGZhy{}y
\PYGZdl{} git clone https://github.com/sqlitebrowser/sqlitebrowser
\PYGZdl{} cd sqlitebrowser
\PYGZdl{} mkdir build
\PYGZdl{} cd build
\PYGZdl{} cmake \PYGZhy{}Dsqlcipher=1 \PYGZhy{}Wno\PYGZhy{}dev ..
\PYGZdl{} make
\PYGZdl{} sudo make install
\end{sphinxVerbatim}

\end{enumerate}

Instructions for installing JDK 8 on Ubuntu 16.04 can be found \sphinxhref{http://www.javahelps.com/2015/03/install-oracle-jdk-in-ubuntu.html}{here}.
After installing JDK 8, run the following command at the terminal to properly configure \sphinxstylestrong{R}

\fvset{hllines={, ,}}%
\begin{sphinxVerbatim}[commandchars=\\\{\}]
\PYGZdl{} sudo R CMD javareconf
\end{sphinxVerbatim}

and then install the \sphinxstylestrong{R} packages (as described above).


\chapter{Pipeline Configuration}
\label{\detokenize{config:pipeline-configuration}}\label{\detokenize{config::doc}}\begin{enumerate}
\item {} 
\sphinxstylestrong{Create the SQLite database}: The openVA pipeline uses an SQLite database to access configuration settings for ODK Aggregate, openVA in R,
and DHIS2. Error and log messages are also stored to this database, along with the VA records downloaded from ODK Aggregate and
the assigned COD.
\begin{enumerate}
\item {} 
The necessary tables and schema are created in the SQL file pipelineDB.sql, which can be downloaded from the
\sphinxhref{https://github.com/D4H-CRVS/OpenVA\_Pipeline/pipelineDB.sql}{OpenVA\_Pipeline GitHub webpage}.  Create the SQLite database in the
same folder as the file \sphinxstyleemphasis{pipeline.py}.

\item {} 
Use SQLCipher to create the pipeline database, assign an encryption key, and populate the database using the following commands
(note that the \sphinxcode{\sphinxupquote{\$}} is the terminal prompt and \sphinxcode{\sphinxupquote{sqlite\textgreater{}}} is the SQLite prompt, i.e., not part of the commands).

\end{enumerate}
\begin{quote}

\fvset{hllines={, ,}}%
\begin{sphinxVerbatim}[commandchars=\\\{\}]
\PYGZdl{} sqlcipher
sqlite\PYGZgt{} .open Pipeline.db
sqlite\PYGZgt{} PRAGMA key=encryption\PYGZus{}key;
sqlite\PYGZgt{} .read \PYGZdq{}pipelineDB.sql\PYGZdq{}
sqlite\PYGZgt{} .tables
sqlite\PYGZgt{} \PYGZhy{}\PYGZhy{} take a look \PYGZhy{}\PYGZhy{}
sqlite\PYGZgt{} .schema ODK\PYGZus{}Conf
sqlite\PYGZgt{} SELECT odkURL from ODK\PYGZus{}Conf;
sqlite\PYGZgt{} .quit
\end{sphinxVerbatim}

Note how the pipeline database is encrypted, and can be accessed via with SQLite command: \sphinxcode{\sphinxupquote{PRAGMA key = "encryption\_key;"}}

\fvset{hllines={, ,}}%
\begin{sphinxVerbatim}[commandchars=\\\{\}]
\PYGZdl{} sqlcipher
sqlite\PYGZgt{} .open Pipeline.db
sqlite\PYGZgt{} .tables

Error: file is encrypted or is not a database

sqlite\PYGZgt{} PRAGMA key = \PYGZdq{}encryption\PYGZus{}key\PYGZdq{};
sqlite\PYGZgt{} .tables
sqlite\PYGZgt{} .quit
\end{sphinxVerbatim}
\end{quote}
\begin{enumerate}
\setcounter{enumii}{2}
\item {} 
Open the file \sphinxstyleemphasis{pipeline.py} and edit the first two lines of code by including the name of the pipeline SQLite database (the default
is \sphinxstyleemphasis{Pipeline.db}) and the encryption password.  The lines in \sphinxstyleemphasis{pipeline.py} are:

\end{enumerate}
\begin{quote}

\fvset{hllines={, ,}}%
\begin{sphinxVerbatim}[commandchars=\\\{\}]
\PYG{c+c1}{\PYGZsh{}\PYGZhy{}\PYGZhy{}\PYGZhy{}\PYGZhy{}\PYGZhy{}\PYGZhy{}\PYGZhy{}\PYGZhy{}\PYGZhy{}\PYGZhy{}\PYGZhy{}\PYGZhy{}\PYGZhy{}\PYGZhy{}\PYGZhy{}\PYGZhy{}\PYGZhy{}\PYGZhy{}\PYGZhy{}\PYGZhy{}\PYGZhy{}\PYGZhy{}\PYGZhy{}\PYGZhy{}\PYGZhy{}\PYGZhy{}\PYGZhy{}\PYGZhy{}\PYGZhy{}\PYGZhy{}\PYGZhy{}\PYGZhy{}\PYGZhy{}\PYGZhy{}\PYGZhy{}\PYGZhy{}\PYGZhy{}\PYGZhy{}\PYGZhy{}\PYGZhy{}\PYGZhy{}\PYGZhy{}\PYGZhy{}\PYGZhy{}\PYGZhy{}\PYGZhy{}\PYGZsh{}}
\PYG{c+c1}{\PYGZsh{} User Settings}
\PYG{n}{sqlitePW} \PYG{o}{=} \PYG{l+s+s2}{\PYGZdq{}}\PYG{l+s+s2}{enilepiP}\PYG{l+s+s2}{\PYGZdq{}}
\PYG{n}{dbName}   \PYG{o}{=} \PYG{l+s+s2}{\PYGZdq{}}\PYG{l+s+s2}{Pipeline.db}\PYG{l+s+s2}{\PYGZdq{}}
\PYG{c+c1}{\PYGZsh{}\PYGZhy{}\PYGZhy{}\PYGZhy{}\PYGZhy{}\PYGZhy{}\PYGZhy{}\PYGZhy{}\PYGZhy{}\PYGZhy{}\PYGZhy{}\PYGZhy{}\PYGZhy{}\PYGZhy{}\PYGZhy{}\PYGZhy{}\PYGZhy{}\PYGZhy{}\PYGZhy{}\PYGZhy{}\PYGZhy{}\PYGZhy{}\PYGZhy{}\PYGZhy{}\PYGZhy{}\PYGZhy{}\PYGZhy{}\PYGZhy{}\PYGZhy{}\PYGZhy{}\PYGZhy{}\PYGZhy{}\PYGZhy{}\PYGZhy{}\PYGZhy{}\PYGZhy{}\PYGZhy{}\PYGZhy{}\PYGZhy{}\PYGZhy{}\PYGZhy{}\PYGZhy{}\PYGZhy{}\PYGZhy{}\PYGZhy{}\PYGZhy{}\PYGZhy{}\PYGZsh{}}
\end{sphinxVerbatim}
\end{quote}

\item {} 
\sphinxstylestrong{Configure Pipeline}: The pipeline connects to ODK Aggregate and DHIS2 servers and thus requires usernames, passwords, and URLs.
Arguments for openVA should also be supplied. We will use
\sphinxhref{https://github.com/sqlitebrowser/sqlitebrowser/blob/master/BUILDING.md}{DB Browser for SQLite} to configure these settings. Start
by launching DB Browser from the terminal, which should open the window below \sphinxcode{\sphinxupquote{\$ sqlitebrowser}}
\begin{quote}

\noindent\sphinxincludegraphics{{dbBrowser}.png}
\end{quote}

Next, open the database by selecting the menu options: \sphinxstyleemphasis{File} -\textgreater{} \sphinxstyleemphasis{Open Database…}
\begin{quote}

\noindent\sphinxincludegraphics{{dbBrowser_open}.png}
\end{quote}

and navigate to the \sphinxstyleemphasis{Pipeline.db} SQLite database and click the \sphinxstyleemphasis{Open} button.  This will prompt you to enter in encryption password.
\begin{quote}

\noindent\sphinxincludegraphics{{dbBrowser_encryption}.png}
\end{quote}
\begin{enumerate}
\item {} 
\sphinxstylestrong{ODK Configuration}: To configure the pipeline connection to ODK Aggregate, click on the \sphinxstyleemphasis{Browse Data} tab and select the
ODK\_Conf table as shown below.
\begin{quote}

\noindent\sphinxincludegraphics{{dbBrowser_browseData}.png}

\noindent\sphinxincludegraphics{{dbBrowser_odk}.png}
\end{quote}

Now, click on the \sphinxstyleemphasis{odkURL} column, enter the URL for your ODK Aggregate server, and click \sphinxstyleemphasis{Apply}.
\begin{quote}

\noindent\sphinxincludegraphics{{dbBrowser_odkURLApply}.png}
\end{quote}

Similarly, edit the \sphinxstyleemphasis{odkUser}, \sphinxstyleemphasis{odkPass}, and \sphinxstyleemphasis{odkFormID} columns so they contain a valid user name, password, and Form ID
(see Form Management on ODK Aggregate server) of the VA questionnaire of your ODK Aggregate server.

\item {} 
\sphinxstylestrong{openVA Configuration}: The pipeline configuration for openVA is stored in the \sphinxstyleemphasis{Pipeline\_Conf} table. Follow the steps described
above (in the ODK Aggregate Configuration section) and edit the following columns:
\begin{itemize}
\item {} 
\sphinxstyleemphasis{workingDirectory} \textendash{} the directory where the pipeline files (i.e., \sphinxstyleemphasis{pipeline.py}, \sphinxstyleemphasis{Pipeline.db} and the ODK Briefcase
application, \sphinxstyleemphasis{ODK-Briefcase-v1.10.1.jar}).  Note that the pipeline will create new folders and files in this working directory,
and must be run by a user with privileges for writing files to this location.

\item {} 
\sphinxstyleemphasis{openVA\_Algorithm} \textendash{} currently, there are only two acceptable values for the alogrithm are \sphinxcode{\sphinxupquote{InterVA}} or \sphinxcode{\sphinxupquote{Insilico}}

\item {} 
\sphinxstyleemphasis{algorithmMetadataCode} \textendash{} this column captures the necessary inputs for producing a COD, namely the VA questionnaire, the
algorithm, and the symptom-cause information (SCI) (see {[}below{]}(\#SCI) for more information on the SCI).  Note that there are also
different versions (e.g., InterVA 4.01 and InterVA 4.02, or WHO 2012 questionnare and the WHO 2016 instrument/questionnaire).  It is
important to keep track of these inputs in order to make the COD determination reproducible and to fully understand the assignment
of the COD.  A list of all algorith metadata codes is provided in the \sphinxstyleemphasis{dhisCode} column in the \sphinxstyleemphasis{Algorithm\_Metadata\_Options} table.
The logic for each code is

algorith\textbar{}algorithm version\textbar{}SCI\textbar{}SCI version\textbar{}instrument\textbar{}instrument version

\item {} 
\sphinxstyleemphasis{codSource} \textendash{} both the InterVA and InSilicoVA algorithms return CODs from a list produced by the WHO, and thus this column should
be left at the default value of \sphinxcode{\sphinxupquote{WHO}}.

\end{itemize}

\item {} 
\sphinxstylestrong{DHIS2 Configuration}: The pipeline configuration for DHIS2 is located in the \sphinxstyleemphasis{DHIS\_Conf} table, and the following columns should
be edited with appropriate values for your DHIS2 server.
\begin{itemize}
\item {} 
\sphinxstyleemphasis{dhisURL} \textendash{}  the URL for your DHIS2 server

\item {} 
\sphinxstyleemphasis{dhisUser} \textendash{} the username for the DHIS2 account

\item {} 
\sphinxstyleemphasis{dhisPass} \textendash{} the password for the DHIS2 account

\item {} 
\sphinxstyleemphasis{dhisOrgUnit} \textendash{} the Organization Unit (e.g., districts) UID to which the verbal autopsies are associated. The organisation unit
must be linked to the Verbal Autopsy program.  For more details, see the DHIS2 Verbal Autopsy program
\sphinxhref{https://github.com/SwissTPH/dhis2\_va\_draft/blob/master/docs/Installation.md}{installation guide}

\end{itemize}

\end{enumerate}

\end{enumerate}


\chapter{Documentation for classes, functions, and methods}
\label{\detokenize{help:module-runPipeline}}\label{\detokenize{help:documentation-for-classes-functions-and-methods}}\label{\detokenize{help::doc}}\index{runPipeline (module)}\phantomsection\label{\detokenize{help:module-pipeline}}\index{pipeline (module)}\phantomsection\label{\detokenize{help:module-transferDB}}\index{transferDB (module)}\phantomsection\label{\detokenize{help:module-odk}}\index{odk (module)}\phantomsection\label{\detokenize{help:module-openVA}}\index{openVA (module)}\phantomsection\label{\detokenize{help:module-dhis}}\index{dhis (module)}

\section{Main Interface}
\label{\detokenize{help:main-interface}}
The OpenVA Pipeline is run using the following function
\index{runPipeline() (in module runPipeline)}

\begin{fulllineitems}
\phantomsection\label{\detokenize{help:runPipeline.runPipeline}}\pysiglinewithargsret{\sphinxcode{\sphinxupquote{runPipeline.}}\sphinxbfcode{\sphinxupquote{runPipeline}}}{\emph{database\_file\_name}, \emph{database\_directory}, \emph{database\_key}, \emph{export\_to\_DHIS=True}}{}
Runs through all steps of the OpenVA Pipeline.
\begin{quote}\begin{description}
\item[{Parameters}] \leavevmode\begin{itemize}
\item {} 
\sphinxstyleliteralstrong{\sphinxupquote{database\_file\_name}} \textendash{} File name for the Transfer Database.

\item {} 
\sphinxstyleliteralstrong{\sphinxupquote{database\_directory}} \textendash{} Path of the Transfer Database.

\item {} 
\sphinxstyleliteralstrong{\sphinxupquote{datatbase\_key}} \textendash{} Encryption key for the Transfer Database

\item {} 
\sphinxstyleliteralstrong{\sphinxupquote{export\_to\_DHIS}} (\sphinxstyleliteralemphasis{\sphinxupquote{(}}\sphinxstyleliteralemphasis{\sphinxupquote{Boolean}}\sphinxstyleliteralemphasis{\sphinxupquote{)}}) \textendash{} Indicator for posting VA records to a DHIS2 server.

\end{itemize}

\end{description}\end{quote}

\end{fulllineitems}



\section{API for Transfer Database}
\label{\detokenize{help:api-for-transfer-database}}\index{TransferDB (class in transferDB)}

\begin{fulllineitems}
\phantomsection\label{\detokenize{help:transferDB.TransferDB}}\pysiglinewithargsret{\sphinxbfcode{\sphinxupquote{class }}\sphinxcode{\sphinxupquote{transferDB.}}\sphinxbfcode{\sphinxupquote{TransferDB}}}{\emph{dbFileName}, \emph{dbDirectory}, \emph{dbKey}, \emph{plRunDate}}{}
This class handles interactions with the Transfer database.

The Pipeline accesses configuration information from the Transfer database,
and also stores log messages and verbal autopsy records in the DB.  The
Transfer database is encrypted using sqlcipher3 (and the pysqlcipher3
module is imported to establish DB connection).
\begin{description}
\item[{dbFileName}] \leavevmode{[}str{]}
File name of the Tranfser database.

\item[{dbDirectory}] \leavevmode{[}str{]}
Path of folder containing the Transfer database.

\item[{dbKey}] \leavevmode{[}str{]}
Encryption key for the Transfer database.

\item[{plRunDate}] \leavevmode{[}date{]}
Date when pipeline started latest run (YYYY-MM-DD\_hh:mm:ss).

\end{description}
\begin{description}
\item[{connectDB(self)}] \leavevmode
Returns SQLite Connection object to Transfer database.

\item[{configPipeline(self, conn)}] \leavevmode
Accepts SQLite Connection object and returns tuple with configuration
settings for the Pipeline.

\item[{configODK(self, conn)}] \leavevmode
Accepts SQLite Connection object and returns tuple with configuration
settings for connecting to ODK Aggregate server.

\item[{configOpenVA(self, conn)}] \leavevmode
Accepts SQLite Connection object and returns tuple with configuration
settings for R package openVA.

\item[{checkDuplicates(self)}] \leavevmode
Search for duplicate VA records in ODK Briefcase export file and the
tranfser DB.

\item[{configDHIS(self, conn)}] \leavevmode
Accepts SQLite Connection object and returns tuple with configuration
settings for connecting to DHIS2 server.

\item[{storeVA(self)}] \leavevmode
Deposits VA objects to the Transfer DB.

\item[{makePipelineDirs(self)}] \leavevmode
Returns SQLite Connection object to Transfer database.

\item[{cleanODK(self)}] \leavevmode
Remove ODK Briefcase Export files.

\item[{cleanOpenVA(self)}] \leavevmode
Remove openVA files with COD results.

\item[{cleanDHIS(self)}] \leavevmode
Remove DHIS2 blob files.

\end{description}
\index{checkDuplicates() (transferDB.TransferDB method)}

\begin{fulllineitems}
\phantomsection\label{\detokenize{help:transferDB.TransferDB.checkDuplicates}}\pysiglinewithargsret{\sphinxbfcode{\sphinxupquote{checkDuplicates}}}{\emph{conn}}{}
Search for duplicate VA records.

This method searches for duplicate VA records in ODK Briefcase export
file and the Tranfser DB.  If duplicates are found, a warning message
is logged to the EventLog table in the Transfer database.

conn : sqlite3 Connection object

DatabaseConnectionError
PipelineError

\end{fulllineitems}

\index{cleanDHIS() (transferDB.TransferDB method)}

\begin{fulllineitems}
\phantomsection\label{\detokenize{help:transferDB.TransferDB.cleanDHIS}}\pysiglinewithargsret{\sphinxbfcode{\sphinxupquote{cleanDHIS}}}{}{}
Remove DHIS2 blob files.

\end{fulllineitems}

\index{cleanODK() (transferDB.TransferDB method)}

\begin{fulllineitems}
\phantomsection\label{\detokenize{help:transferDB.TransferDB.cleanODK}}\pysiglinewithargsret{\sphinxbfcode{\sphinxupquote{cleanODK}}}{}{}
Remove ODK Briefcase Export files.

\end{fulllineitems}

\index{cleanOpenVA() (transferDB.TransferDB method)}

\begin{fulllineitems}
\phantomsection\label{\detokenize{help:transferDB.TransferDB.cleanOpenVA}}\pysiglinewithargsret{\sphinxbfcode{\sphinxupquote{cleanOpenVA}}}{}{}
Remove openVA files with COD results.

\end{fulllineitems}

\index{configDHIS() (transferDB.TransferDB method)}

\begin{fulllineitems}
\phantomsection\label{\detokenize{help:transferDB.TransferDB.configDHIS}}\pysiglinewithargsret{\sphinxbfcode{\sphinxupquote{configDHIS}}}{\emph{conn}, \emph{algorithm}}{}
Query DHIS configuration settings from database.

This method is intended to be used in conjunction with (1)
TransferDB.connectDB(), which establishes a connection to a database
with the Pipeline configuration settings; and (2) DHIS.connect(), which
establishes a connection to a DHIS server.  Thus,
TransferDB.configDHIS() gets its input from TransferDB.connectDB() and
the output from TransferDB.config() is a valid argument for
DHIS.config().
\begin{description}
\item[{conn}] \leavevmode{[}sqlite3 Connection object (e.g., the object returned from{]}
TransferDB.connectDB())

\end{description}

algorithm : VA algorithm used by R package openVA
\begin{description}
\item[{tuple}] \leavevmode
Contains all parameters for DHIS.connect().

\end{description}

DHISConfigurationError

\end{fulllineitems}

\index{configODK() (transferDB.TransferDB method)}

\begin{fulllineitems}
\phantomsection\label{\detokenize{help:transferDB.TransferDB.configODK}}\pysiglinewithargsret{\sphinxbfcode{\sphinxupquote{configODK}}}{\emph{conn}}{}
Query ODK configuration settings from database.

This method is intended to be used in conjunction with (1)
TransferDB.connectDB(), which establishes a connection to a database
with the Pipeline configuration settings; and (2) ODK.briefcase(), which
establishes a connection to an ODK Aggregate server.  Thus,
TransferDB.configODK() gets its input from TransferDB.connectDB() and
the output from TransferDB.configODK() is a valid argument for ODK.config().

conn : sqlite3 Connection object
\begin{description}
\item[{tuple}] \leavevmode
Contains all parameters for ODK.briefcase().

\end{description}

ODKConfigurationError

\end{fulllineitems}

\index{configOpenVA() (transferDB.TransferDB method)}

\begin{fulllineitems}
\phantomsection\label{\detokenize{help:transferDB.TransferDB.configOpenVA}}\pysiglinewithargsret{\sphinxbfcode{\sphinxupquote{configOpenVA}}}{\emph{conn}, \emph{algorithm}, \emph{pipelineDir}}{}
Query OpenVA configuration settings from database.

This method is intended to receive its input (a Connection object) 
from TransferDB.connectDB(), which establishes a connection to a
database with the Pipeline configuration settings.  It sets up the
configuration for all of the VA algorithms included in the R package
openVA.  The output from configOpenVA() serves as an input to the
method OpenVA.setAlgorithmParameters().  This is a wrapper function
that calls \_\_configInterVA\_\_, \_\_configInSilicoVA\_\_, and
\_\_configSmartVA\_\_ to actually pull configuration settings from the
database.

conn : sqlite3 Connection object
algorithm : VA algorithm used by R package openVA
pipelineDir : Working directory for the Pipeline
\begin{description}
\item[{tuple}] \leavevmode
Contains all parameters needed for OpenVA.setAlgorithmParameters().

\end{description}

OpenVAConfigurationError

\end{fulllineitems}

\index{configPipeline() (transferDB.TransferDB method)}

\begin{fulllineitems}
\phantomsection\label{\detokenize{help:transferDB.TransferDB.configPipeline}}\pysiglinewithargsret{\sphinxbfcode{\sphinxupquote{configPipeline}}}{\emph{conn}}{}
Grabs Pipline configuration settings.

This method queries the Pipeline\_Conf table in Transfer database and
returns a tuple with attributes (1) algorithmMetadataCode; (2)
codSource; (3) algorithm; and (4) workingDirectory.
\begin{description}
\item[{tuple}] \leavevmode
alogrithmMetadataCode - attribute describing VA data
codSource - attribute detailing the source of the Cause of Death list
algorithm - attribute indicating which VA algorithm to use
workingDirectory - attribute indicating the working directory

\end{description}

PipelineConfigurationError

\end{fulllineitems}

\index{connectDB() (transferDB.TransferDB method)}

\begin{fulllineitems}
\phantomsection\label{\detokenize{help:transferDB.TransferDB.connectDB}}\pysiglinewithargsret{\sphinxbfcode{\sphinxupquote{connectDB}}}{}{}
Connect to Transfer database.

Uses parameters supplied to the parent class, TransferDB, to connect to
the (encrypted) Transfer database.
\begin{description}
\item[{SQLite database connection object}] \leavevmode
Used to query (encrypted) SQLite database.

\end{description}

DatabaseConnectionError

\end{fulllineitems}

\index{makePipelineDirs() (transferDB.TransferDB method)}

\begin{fulllineitems}
\phantomsection\label{\detokenize{help:transferDB.TransferDB.makePipelineDirs}}\pysiglinewithargsret{\sphinxbfcode{\sphinxupquote{makePipelineDirs}}}{}{}
Create directories for storing files (if they don’t exist).

The method creates the following folders in the working directory (as
set in the Transfer database table Pipeline\_Conf): (1) ODKFiles for
files containing verbal autopsy records from the ODK Aggregate server; (2)
OpenVAFiles containing R scripts and results from the cause assignment
algorithms; and (3) DHIS for holding blobs that will be stored in a
data repository (DHIS2 server and/or the local Transfer database).

PipelinError

\end{fulllineitems}

\index{storeVA() (transferDB.TransferDB method)}

\begin{fulllineitems}
\phantomsection\label{\detokenize{help:transferDB.TransferDB.storeVA}}\pysiglinewithargsret{\sphinxbfcode{\sphinxupquote{storeVA}}}{\emph{conn}}{}
Store VA records in Transfer database.

This method is intended to be used in conjunction with the DHIS module,
which prepares the records into the proper format for storage in the
Transfer database.

conn : sqlite3 Connection object

PipelineError
DatabaseConnectionError

\end{fulllineitems}

\index{updateODKLastRun() (transferDB.TransferDB method)}

\begin{fulllineitems}
\phantomsection\label{\detokenize{help:transferDB.TransferDB.updateODKLastRun}}\pysiglinewithargsret{\sphinxbfcode{\sphinxupquote{updateODKLastRun}}}{\emph{conn}, \emph{plRunDate}}{}
Update Transfer Database table ODK\_Conf.odkLastRun

conn : sqlite3 Connection object
plRunDate : date
Date when pipeline started latest run (YYYY-MM-DD\_hh:mm:ss)

\end{fulllineitems}


\end{fulllineitems}



\section{API for ODK Briefcase}
\label{\detokenize{help:api-for-odk-briefcase}}\index{ODK (class in odk)}

\begin{fulllineitems}
\phantomsection\label{\detokenize{help:odk.ODK}}\pysiglinewithargsret{\sphinxbfcode{\sphinxupquote{class }}\sphinxcode{\sphinxupquote{odk.}}\sphinxbfcode{\sphinxupquote{ODK}}}{\emph{odkSettings}, \emph{workingDirectory}}{}
Manages Pipeline’s interaction with ODK Aggregate.

This class handles the segment of the pipeline related to ODK.  The
ODK.connect() method calls ODK Briefcase to connect with an ODK Aggregate
server and export VA records.  It also checks for previously exported files
and updates them as needed.  Finally, it logs messages and errors to the
pipeline database.
\begin{description}
\item[{odkSettings}] \leavevmode{[}(named) tuple with all of configuration settings as{]}
attributes.

\end{description}

workingDirectory : Directory where openVA Pipeline should create files.
\begin{description}
\item[{briefcase(self)}] \leavevmode
Uses ODK Briefcase to export VA records from ODK Aggregate server.

\item[{mergePrevExport(self)}] \leavevmode
Merge ODK Briefcase export files (if they exist).

\end{description}

ODKBriefcaseError
\index{briefcase() (odk.ODK method)}

\begin{fulllineitems}
\phantomsection\label{\detokenize{help:odk.ODK.briefcase}}\pysiglinewithargsret{\sphinxbfcode{\sphinxupquote{briefcase}}}{}{}
Export records from ODK Aggregate server using ODK Briefcase.

Longer description here.
\begin{description}
\item[{Connection object}] \leavevmode
SQL connection object for querying config settings

\end{description}

ODKError

\end{fulllineitems}

\index{mergeToPrevExport() (odk.ODK method)}

\begin{fulllineitems}
\phantomsection\label{\detokenize{help:odk.ODK.mergeToPrevExport}}\pysiglinewithargsret{\sphinxbfcode{\sphinxupquote{mergeToPrevExport}}}{}{}
Merge previous ODK Briefcase export files.

\end{fulllineitems}


\end{fulllineitems}



\section{API for OpenVA}
\label{\detokenize{help:api-for-openva}}\index{OpenVA (class in openVA)}

\begin{fulllineitems}
\phantomsection\label{\detokenize{help:openVA.OpenVA}}\pysiglinewithargsret{\sphinxbfcode{\sphinxupquote{class }}\sphinxcode{\sphinxupquote{openVA.}}\sphinxbfcode{\sphinxupquote{OpenVA}}}{\emph{vaArgs}, \emph{pipelineArgs}, \emph{odkID}, \emph{runDate}}{}
Assign cause of death (COD) to verbal autopsies (VA) R package openVA.

This class creates and executes an R script that copies (and merges) ODK Briefcase
exports, runs openVA to assign CODs, and creates outputs for depositing in
the Transfers DB and to a DHIS server.
\begin{description}
\item[{algorithm}] \leavevmode{[}str{]}
Which VA algorithm should be used to assign COD.

\end{description}
\begin{description}
\item[{copyVA(self)}] \leavevmode
Create data file for openVA by merging ODK export files.

\item[{rScript(self)}] \leavevmode
Wrapper for algorithm-specific methods that create an R script to use
openVA to assign CODs.

\item[{\_\_rScript\_InSilicoVA(self)\_\_}] \leavevmode
Write an R script for running InSilicoVA and assigning CODs.

\item[{\_\_rScript\_InterVA(self)\_\_}] \leavevmode
Write an R script for running InterVA and assigning CODs.

\item[{getCOD(self)}] \leavevmode
Run R as subprocess and run the script for assigning CODs.

\end{description}

OpenVAError
\index{copyVA() (openVA.OpenVA method)}

\begin{fulllineitems}
\phantomsection\label{\detokenize{help:openVA.OpenVA.copyVA}}\pysiglinewithargsret{\sphinxbfcode{\sphinxupquote{copyVA}}}{}{}
Create data file for openVA by merging ODK export files.

\end{fulllineitems}

\index{getCOD() (openVA.OpenVA method)}

\begin{fulllineitems}
\phantomsection\label{\detokenize{help:openVA.OpenVA.getCOD}}\pysiglinewithargsret{\sphinxbfcode{\sphinxupquote{getCOD}}}{}{}
Create and execute R script to assign a COD with openVA; or call the SmartVA CLI to assign COD.

\end{fulllineitems}

\index{rScript() (openVA.OpenVA method)}

\begin{fulllineitems}
\phantomsection\label{\detokenize{help:openVA.OpenVA.rScript}}\pysiglinewithargsret{\sphinxbfcode{\sphinxupquote{rScript}}}{}{}
Create an R script for running openVA and assigning CODs.

\end{fulllineitems}

\index{smartVA\_to\_csv() (openVA.OpenVA method)}

\begin{fulllineitems}
\phantomsection\label{\detokenize{help:openVA.OpenVA.smartVA_to_csv}}\pysiglinewithargsret{\sphinxbfcode{\sphinxupquote{smartVA\_to\_csv}}}{}{}~\begin{description}
\item[{Write two CSV files: (1) Entity Value Attribute blob pushed to DHIS2 (entityAttributeValue.csv)}] \leavevmode\begin{enumerate}
\setcounter{enumi}{1}
\item {} 
table for transfer database (recordStorage.csv)

\end{enumerate}

\end{description}

Both CSV files are stored in the OpenVA folder.

\end{fulllineitems}


\end{fulllineitems}



\section{API for DHIS2}
\label{\detokenize{help:api-for-dhis2}}\index{DHIS (class in dhis)}

\begin{fulllineitems}
\phantomsection\label{\detokenize{help:dhis.DHIS}}\pysiglinewithargsret{\sphinxbfcode{\sphinxupquote{class }}\sphinxcode{\sphinxupquote{dhis.}}\sphinxbfcode{\sphinxupquote{DHIS}}}{\emph{dhisArgs}, \emph{workingDirectory}}{}
Class for transfering VA records (with assigned CODs) to the DHIf2S server.

This class includes methods for importing VA results (i.e. assigned causes of
death from openVA or SmartVA) as CSV files, connecting to a DHIS2 server
with the Verbal Autopsy Program, and posting the results to the DHIS2
server and/or the local Transfer database.
\begin{description}
\item[{dhisArgs}] \leavevmode{[}(named) tuple{]}
Contains parameter values for connected to DHIS2, as returned by
transferDB.configDHIS().

\end{description}
\begin{description}
\item[{connect(self)}] \leavevmode
Wrapper for algorithm-specific methods that create an R script to use
openVA to assign CODs.

\item[{postVA(self)}] \leavevmode
Prepare and post VA objects to the DHIS2 server.

\item[{verifyPost(self)}] \leavevmode
Verify that VA records were posted to DHIS2 server.

\item[{CheckDuplicates(self)}] \leavevmode
Checks the DHIS2 server for duplicate records.

\end{description}

DHISError
\index{connect() (dhis.DHIS method)}

\begin{fulllineitems}
\phantomsection\label{\detokenize{help:dhis.DHIS.connect}}\pysiglinewithargsret{\sphinxbfcode{\sphinxupquote{connect}}}{}{}
Run get method to retrieve VA program ID and Org Unit.

\end{fulllineitems}

\index{postVA() (dhis.DHIS method)}

\begin{fulllineitems}
\phantomsection\label{\detokenize{help:dhis.DHIS.postVA}}\pysiglinewithargsret{\sphinxbfcode{\sphinxupquote{postVA}}}{\emph{apiDHIS}}{}
Post VA records to DHIS.

\end{fulllineitems}

\index{verifyPost() (dhis.DHIS method)}

\begin{fulllineitems}
\phantomsection\label{\detokenize{help:dhis.DHIS.verifyPost}}\pysiglinewithargsret{\sphinxbfcode{\sphinxupquote{verifyPost}}}{\emph{postLog}, \emph{apiDHIS}}{}
Verify that VA records were posted to DHIS2 server.

\end{fulllineitems}


\end{fulllineitems}

\index{API (class in dhis)}

\begin{fulllineitems}
\phantomsection\label{\detokenize{help:dhis.API}}\pysiglinewithargsret{\sphinxbfcode{\sphinxupquote{class }}\sphinxcode{\sphinxupquote{dhis.}}\sphinxbfcode{\sphinxupquote{API}}}{\emph{dhisURL}, \emph{dhisUser}, \emph{dhisPass}}{}
This class provides methods for interacting with the DHIS2 API.
\begin{description}
\item[{dhisURL}] \leavevmode{[}str{]}
Web address for DHIS2 server (e.g., “play.dhis2.org/demo”).

\item[{dhisUser}] \leavevmode{[}str{]}
Username for DHIS2 account.

\item[{dhisPassword}] \leavevmode{[}str{]}
Password for DHIS2 account.

\end{description}
\begin{description}
\item[{get(self, endpoint, params)}] \leavevmode
GET method for DHIS2 API.

\item[{post(self, endpoint, data) }] \leavevmode
POST method for DIHS2 API.

\item[{post\_blob(self, f) }] \leavevmode
Post file to DHIS2 and return created UID for that file.

\end{description}

DHISError
\index{get() (dhis.API method)}

\begin{fulllineitems}
\phantomsection\label{\detokenize{help:dhis.API.get}}\pysiglinewithargsret{\sphinxbfcode{\sphinxupquote{get}}}{\emph{endpoint}, \emph{params=None}}{}
GET method for DHIS2 API.
:rtype: dict

\end{fulllineitems}

\index{post() (dhis.API method)}

\begin{fulllineitems}
\phantomsection\label{\detokenize{help:dhis.API.post}}\pysiglinewithargsret{\sphinxbfcode{\sphinxupquote{post}}}{\emph{endpoint}, \emph{data}}{}
POST  method for DHIS2 API.
:rtype: dict

\end{fulllineitems}

\index{post\_blob() (dhis.API method)}

\begin{fulllineitems}
\phantomsection\label{\detokenize{help:dhis.API.post_blob}}\pysiglinewithargsret{\sphinxbfcode{\sphinxupquote{post\_blob}}}{\emph{f}}{}
Post file to DHIS2 and return created UID for that file
:rtype: str

\end{fulllineitems}


\end{fulllineitems}

\index{VerbalAutopsyEvent (class in dhis)}

\begin{fulllineitems}
\phantomsection\label{\detokenize{help:dhis.VerbalAutopsyEvent}}\pysiglinewithargsret{\sphinxbfcode{\sphinxupquote{class }}\sphinxcode{\sphinxupquote{dhis.}}\sphinxbfcode{\sphinxupquote{VerbalAutopsyEvent}}}{\emph{va\_id}, \emph{program}, \emph{dhis\_orgunit}, \emph{event\_date}, \emph{sex}, \emph{dob}, \emph{age}, \emph{cod\_code}, \emph{algorithm\_metadata}, \emph{odk\_id}, \emph{file\_id}}{}
DHIS2 event + a BLOB file resource
\index{format\_to\_dhis2() (dhis.VerbalAutopsyEvent method)}

\begin{fulllineitems}
\phantomsection\label{\detokenize{help:dhis.VerbalAutopsyEvent.format_to_dhis2}}\pysiglinewithargsret{\sphinxbfcode{\sphinxupquote{format\_to\_dhis2}}}{\emph{dhisUser}}{}
Format object to DHIS2 compatible event for DHIS2 API
:rtype: dict

\end{fulllineitems}


\end{fulllineitems}

\index{create\_db() (in module dhis)}

\begin{fulllineitems}
\phantomsection\label{\detokenize{help:dhis.create_db}}\pysiglinewithargsret{\sphinxcode{\sphinxupquote{dhis.}}\sphinxbfcode{\sphinxupquote{create\_db}}}{\emph{fName}, \emph{evaList}}{}
Create a SQLite database with VA data + COD
:rtype: None

\end{fulllineitems}

\index{getCODCode() (in module dhis)}

\begin{fulllineitems}
\phantomsection\label{\detokenize{help:dhis.getCODCode}}\pysiglinewithargsret{\sphinxcode{\sphinxupquote{dhis.}}\sphinxbfcode{\sphinxupquote{getCODCode}}}{\emph{myDict}, \emph{searchFor}}{}
Return COD label expected by DHIS2.
:rtype: str

\end{fulllineitems}

\index{findKeyValue() (in module dhis)}

\begin{fulllineitems}
\phantomsection\label{\detokenize{help:dhis.findKeyValue}}\pysiglinewithargsret{\sphinxcode{\sphinxupquote{dhis.}}\sphinxbfcode{\sphinxupquote{findKeyValue}}}{\emph{key}, \emph{d}}{}
Return a key’s value in a nested dictionary.

\end{fulllineitems}



\section{Excetptions}
\label{\detokenize{help:excetptions}}\index{PipelineError}

\begin{fulllineitems}
\phantomsection\label{\detokenize{help:transferDB.PipelineError}}\pysigline{\sphinxbfcode{\sphinxupquote{exception }}\sphinxcode{\sphinxupquote{transferDB.}}\sphinxbfcode{\sphinxupquote{PipelineError}}}
Base class for exceptions in the openva\_pipeline module.

\end{fulllineitems}

\index{DatabaseConnectionError}

\begin{fulllineitems}
\phantomsection\label{\detokenize{help:transferDB.DatabaseConnectionError}}\pysigline{\sphinxbfcode{\sphinxupquote{exception }}\sphinxcode{\sphinxupquote{transferDB.}}\sphinxbfcode{\sphinxupquote{DatabaseConnectionError}}}
\end{fulllineitems}

\index{ODKConfigurationError}

\begin{fulllineitems}
\phantomsection\label{\detokenize{help:transferDB.ODKConfigurationError}}\pysigline{\sphinxbfcode{\sphinxupquote{exception }}\sphinxcode{\sphinxupquote{transferDB.}}\sphinxbfcode{\sphinxupquote{ODKConfigurationError}}}
\end{fulllineitems}

\index{OpenVAConfigurationError}

\begin{fulllineitems}
\phantomsection\label{\detokenize{help:transferDB.OpenVAConfigurationError}}\pysigline{\sphinxbfcode{\sphinxupquote{exception }}\sphinxcode{\sphinxupquote{transferDB.}}\sphinxbfcode{\sphinxupquote{OpenVAConfigurationError}}}
\end{fulllineitems}

\index{DHISConfigurationError}

\begin{fulllineitems}
\phantomsection\label{\detokenize{help:transferDB.DHISConfigurationError}}\pysigline{\sphinxbfcode{\sphinxupquote{exception }}\sphinxcode{\sphinxupquote{transferDB.}}\sphinxbfcode{\sphinxupquote{DHISConfigurationError}}}
\end{fulllineitems}

\index{ODKError}

\begin{fulllineitems}
\phantomsection\label{\detokenize{help:odk.ODKError}}\pysigline{\sphinxbfcode{\sphinxupquote{exception }}\sphinxcode{\sphinxupquote{odk.}}\sphinxbfcode{\sphinxupquote{ODKError}}}
\end{fulllineitems}

\index{OpenVAError}

\begin{fulllineitems}
\phantomsection\label{\detokenize{help:openVA.OpenVAError}}\pysigline{\sphinxbfcode{\sphinxupquote{exception }}\sphinxcode{\sphinxupquote{openVA.}}\sphinxbfcode{\sphinxupquote{OpenVAError}}}
\end{fulllineitems}

\index{SmartVAError}

\begin{fulllineitems}
\phantomsection\label{\detokenize{help:openVA.SmartVAError}}\pysigline{\sphinxbfcode{\sphinxupquote{exception }}\sphinxcode{\sphinxupquote{openVA.}}\sphinxbfcode{\sphinxupquote{SmartVAError}}}
\end{fulllineitems}



\chapter{Indices and tables}
\label{\detokenize{index:indices-and-tables}}\begin{itemize}
\item {} 
\DUrole{xref,std,std-ref}{genindex}

\item {} 
\DUrole{xref,std,std-ref}{modindex}

\item {} 
\DUrole{xref,std,std-ref}{search}

\end{itemize}


\renewcommand{\indexname}{Python Module Index}
\begin{sphinxtheindex}
\def\bigletter#1{{\Large\sffamily#1}\nopagebreak\vspace{1mm}}
\bigletter{d}
\item {\sphinxstyleindexentry{dhis}}\sphinxstyleindexpageref{help:\detokenize{module-dhis}}
\indexspace
\bigletter{o}
\item {\sphinxstyleindexentry{odk}}\sphinxstyleindexpageref{help:\detokenize{module-odk}}
\item {\sphinxstyleindexentry{openVA}}\sphinxstyleindexpageref{help:\detokenize{module-openVA}}
\indexspace
\bigletter{p}
\item {\sphinxstyleindexentry{pipeline}}\sphinxstyleindexpageref{help:\detokenize{module-pipeline}}
\indexspace
\bigletter{r}
\item {\sphinxstyleindexentry{runPipeline}}\sphinxstyleindexpageref{help:\detokenize{module-runPipeline}}
\indexspace
\bigletter{t}
\item {\sphinxstyleindexentry{transferDB}}\sphinxstyleindexpageref{help:\detokenize{module-transferDB}}
\end{sphinxtheindex}

\renewcommand{\indexname}{Index}
\printindex
\end{document}